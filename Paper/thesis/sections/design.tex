% !TEX root = ../thesis.tex

\chapter{Design}
The design and development of the game will be handled in the second semester of this academic year. To make sure the second phase of the thesis goes as smooth as possible, it is still handy to plan ahead and think about the technologies that will be used and the features offered by the final product.

\section{Mixed Reality Robotic Game}
In the final game, a physical robot will be controlled by a user with a HoloLens. Some virtual objects will be hovering over the robot and the user has to steer the robot using eye gaze to shoot them down. To add to the user's gameplay experience, some must-have game elements like high-scores, player upgrades or a difficulty setting will be added.

\begin{figure}[!htb]
	\includegraphics[width=1.0\textwidth]{images/MRmockup.PNG}
	\captionsetup{width=1.0\textwidth}
	\centering
	\caption{This mockup gives an idea on what the final product should look like. We can see the user controlling the Lego robot using the Microsoft HoloLens. Virtual agents ---in this case bees--- follow the robot around and can be shot down by the user using the robot's firing system. For every hit, the player's score increases.}
\end{figure}

\section{Implementation Details}
A range of technologies will be used for the development of the game. We will start using these in the second semester. In the meantime we cite a brief overview.

\paragraph{Unity as 3D Engine}
Unity\footnote{\protect\url{https://en.wikipedia.org/wiki/Unity\_(game\_engine)}} is a cross-platform game engine that can be used to implement both 2D and 3D games. It is widely used by amateur game enthusiasts and boosts an excellent Integrated Development Environment (IDE) that makes the whole coding experience easy and enjoyable. 

\paragraph{Programming language: C\#} 
C\#\footnote{\protect\url{https://en.wikipedia.org/wiki/C\_Sharp\_(programming\_language)}} is the default programming language for the Unity game engine. To make coding go as smooth as possible, it speaks for itself that we will be using it.

\paragraph{Microsoft HoloLens}
The Microsoft HoloLens is one of the first commercially available head-mounted devices that is portable enough to be used for this thesis. We will use it as the user's head-mounted device for path specification and directing interaction between the robot and virtual agents.

